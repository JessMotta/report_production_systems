\chapter{O CONTROLE ESTATÍSTICO DE PROCESSOS}
\label{chap:controle_estatistico_de_processos}


%------------------------------------------------------------------

\section{Ferramentas}
\label{sec:controle_estatistico_sec1}

\subsection{Diagrama de Correlação}
É utilizado de forma a verificar a relação dos problemas/defeitos encontrados com relação ao tempo (correlação temporal) ou relação entre problemas e suas possíveis causas (correlação causal). É uma ferramenta onde os dados são transformados em informações úteis ao direcionamento das análises de problemas pelo pessoal da linha de frente.

\subsection{Histogramas}
É uma ferramenta que apresenta os dados de forma gráfica de forma a simplificar a comparação de suas frequências de ocorrência. Muitas vezes o histograma pode ser gerado no chão de fábrica durante a produção.

\subsection{Cartas de Controle de Processos}
Essa ferramenta tem o objetivo de manter o controle de um processo através do acompanhamento do comportamento de uma ou várias variáveis importantes deste processo.

\subsection{Folhas de Verificação}
Esta ferramenta é utilizadade de forma a não se esquecer de empregar as ferramentas citadas anteriormente. As folhas de verificação são conhecidas também como Checklist e são aplicadas para verificação de procedimentos de segurança, atividades e cumprimento das etapas do processo produtivo.




%------------------------------------------------------------------
\section{Aplicação Prática}
\label{sec:controle_estatistico_aplicacao}
