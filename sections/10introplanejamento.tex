\chapter{Introdução ao planejamento e controle da produção}
\label{chap:introducao_ao_planejamento}

Nesta seção mostram-se algumas definições e conceitos introdutórios sobre o \ac{PCP}, em especial, os métodos qualitativos e quantitativos de previsão de demanda a curto, médio e longo prazo. Para isso, primeiramente é necessário entender e distinguir os tipos de demandas existentes no mercado que serão vistos na seção a seguir.

\section{Tipos de demanda}
\label{sec:introducao_ao_planejamento_sec1}

O trabalho do \ac{PCP} tem como objetivo compatibilizar a capacidade de produção da empresa com a demanda a ser atendida, consequentemente, a demanda é quem dita a natureza do atendimento. Existem dois tipos básico de demanda: a independente e a dependente e suas características serão descritas a seguir.

\subsection{Demanda independente}

Este é o tipo mais frenquentemente utilizado pela grande maioria das empresas. De fato, este tipo de demanda ocorre quando as empresas fazem previsões sobre as quantidades de produtos/serviços que seus usuários comprarão em um período de tempo próximo, portanto, preparam-se para aquele atendimento na esperança de que a demanda será bem próxima da previsão. A demanda independente é comum em lojas de varejo, empresas prestadores de serviços pessoais e grande parte das indústrias de bens de consumo final.

\subsection{Demanda dependente}

A demanda dependente só ocorre em situações particulares e menos frequentes. Este tipo de demanda surge quando as empresas possuem, de antemão, as informações de que necessita para elaborar seus planos de produção de curto prazo, ou seja, ela não se atenta em fazer previsões ou estimativas sobre as quantidades de produtos/serviços que seus clientes adquirirão em um curto período de tempo. Essa demanda surge quando as empresas têm o conhecimento prévio da intenção do cliente, está associada a um evento certo e quando há pedidos confirmados.

\section{Métodos de previsão de demanda}
%! todo Não sei se ta certo

Existem dois métodos para tratar as informações disponíveis das previsões de curto, médio e longo prazo que são as abordagens quantitativas e as abordagens qualitativas. Vale salientar que qualquer processo de previsão, no geral, vai conter tanto considerações de natureza mais qualitativa como considerações mais quantitativas a respeito dos dados disponíveis. As características destas duas técnicas serão descritas a seguir

\subsection{Método qualitativo}

Esse método é subjetivo pois reúne os fatores de julgamento e intuição das análises dos dados disponíveis. Estes fatores são úteis quando se espera que esses dados, mais subjetivos, possam ter a capacidade de explicar o futuro, ou quando os dados quantitativos precisos e completos são muito caros ou difíceis de serem obtidos. Existem 5 métodos qualitativos para a precisão de demanda:

\textbf{Método Delphi:} é um processo interativo que permite aos especialistas, mesmo distantes uns dos outros, reunir seus palpites ao processo de previsão. O objetivo deste processo é evitar que uma ou poucas opiniões predominem por fatores exógenos ao propósito de gerar boas previsões.

\textbf{Júri de Executivos:} método em que se captura a opinião de pequenos grupos, em geral, de executivos de nível alto sobre alguma variável que se pretenda prever.

\textbf{Força de vendas:} tal método consiste em unir as estimativas localizadas e desagregadas emitidas por cada vendedor, ou representante de força de vendas, e com isso será formado uma estimativa global.

\textbf{Pesquisa de mercado:} esse método consiste em solicitar diretamente dos possíveis clientes ou consumidores sua intenção de compra futura.

\textbf{Analogia histórica:} método que procura identificar produtos similares dos quais possuem dados para, por analogia, melhor estimar. %? estimar o que a demanada?

\subsection{Método quantitativo}

Esse método de previsão de demanda é baseado na análise das séries de dados históricos em que se busca identificar os padrões de comportamento, para que estes sejam então projetados para o futuro. Ou seja, a previsão do futuro é baseado em dados do passado.

\section{Aplicação prática}
\label{sec:introducao_ao_planejamento_aplicacao}
A estrutura do parque permite fornecer até 60 MW, estipulado por contrato com a Chesf. A demanda é dependente por contrato firmado entre o cliente e a SunBurn. Contudo, a demanda de energia elétrica oscila durante o dia a dia, assim, deve-ser ter uma regulação prevista de acordo com dados históricos para a operação diária, caracterizando-se como o método quantitativo.
