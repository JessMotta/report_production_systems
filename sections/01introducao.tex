\chapter{INTRODUÇÃO}
\label{chap:intro}
Sistemas Produtivos são os processos utilizados na manufatura de produtos e serviços. Desta forma, o seu estudo sistematizado é de fundamental importância para o projeto, construção e gerenciamento de unidades produtivas. Por unidades produtivas, entenda-se qualquer empreendimento que gere um bem ou serviço a ser comercializado para um cliente. Bens podem ser tanto tangíveis ou intangíveis. Bens tangíveis podem ser simplificadamente entendidos como aqueles que possuem um corpo físico mensurável, o que não ocorre nos bens intangíveis.

Assim, as principais características, modelos e abordagens utilizadas no desenvolvimento e gerenciamento de Sistemas Produtivos foram analisadas durante a disciplina de Sistemas Produtivos ministrada pelo Prof. Dr. Francisco Uchoa Passos durante o segundo semestre de 2020 junto ao SENAI Cimatec. Este estudo foi realizado através da apresentação de seminários e investigação da utilização por uma empresa modelo dos conhecimentos debatidos.


%------------------------------------------------------------------
\section{Objetivos}
\label{sec:obj}
Este trabalho tem como objetivo principal apresentar como as teorias e conteúdos estudados durante a disciplina de Sistemas Produtivas são aplicados na empresa modelo escolhida.


%------------------------------------------------------------------
\section{Organização do relatório}
\label{sec:org}
Este documento está organizado da seguinte forma, o capítulo atual faz uma disposição introdutória sobre o trabalho realizado. Os capítulos subsequentes apresentam de forma sucinta os conteúdos e teorias estudados, bem como, as aplicações destes na empresa modelo. Estes estão organizados de acordo com a separação temática apresentada nos materiais de leitura fornecidos.

%----------------------- marcado
\section{Resumo da empresa}
\label{sec:rese}
A SunBurn, nome fictício, é uma empresa que atua no desenvolvimento, implantação e operação de projetos de energia renovável. No Brasil, é sediada no sul do país e opera nas regiões Norte, Sul e Nordeste.

Os projetos da empresa, nos Ambientes de Contratação Regulada (ACR) e Contratação Livre (ACL), somam 642 Megawatts de potência vendida. Todos os empreendimentos são monitorados à distância por meio do Centro de Operações localizado na sede da SunBurn, na região Sul.
A SunBurn estabelece um modelo de negócios com maior segurança e rentabilidade a seus investidores, mantendo o compromisso de fornecer energia limpa e confiável.

Os empreendimentos têm como característica fundamental a qualidade, apresentando altos fatores de capacidade e geração garantida. Aliado ao modelo de gestão da SunBurn, que segue os princípios do ESG (Environmental, Social and Corporate Governance), a alta tecnologia e profissionais qualificados garantem confiabilidade na operação.

A sustentabilidade é fator indissociável da estratégia de negócios da SunBurn. Nas regiões onde a empresa atua, as operações têm foco na redução de impactos ambientais, no desenvolvimento das comunidades da região e na segurança dos colaboradores.