\chapter{INTRODUÇÃO}
\label{chap:intro}

%------------------------------------------------------------------
\section{Objetivos}
\label{sec:obj}


%------------------------------------------------------------------
\section{Justificativa} %motivação
\label{sec:just}
Nesta etapa da automação de um manipulador, 


%------------------------------------------------------------------
\section{Organização do relatório}
\label{sec:org}
Este documento está organizado da seguinte forma, o capítulo 

%----------------------- marcado
\section{Resumo da empresa}
\label{sec:rese}
A SunBurn é uma empresa de nome fictício que atua no desenvolvimento, implantação e operação de projetos de energia renovável. No Brasil, é sediada no sul do país e opera nas regiões Norte, Sul e Nordeste.

Os projetos da empresa, nos Ambientes de Contratação Regulada (ACR) e Contratação Livre (ACL), somam 642 Megawatts de potência vendida. Todos os empreendimentos são monitorados à distância por meio do Centro de Operações localizado na sede da SunBurn, na região Sul.
A SunBurn estabelece um modelo de negócios com maior segurança e rentabilidade a seus investidores, mantendo o compromisso de fornecer energia limpa e confiável.

Os empreendimentos têm como característica fundamental a qualidade, apresentando altos fatores de capacidade e geração garantida. Aliado ao modelo de gestão da SunBurn, que segue os princípios do ESG (Environmental, Social and Corporate Governance), a alta tecnologia e profissionais qualificados garantem confiabilidade na operação.

A sustentabilidade é fator indissociável da estratégia de negócios da SunBurn. Nas regiões onde a empresa atua, as operações têm foco na redução de impactos ambientais, no desenvolvimento das comunidades da região e na segurança dos colaboradores.




