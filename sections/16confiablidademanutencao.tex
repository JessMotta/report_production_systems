\chapter{CONFIABILIDADE E MANUTENÇÃO DO SISTEMA PRODUTIVO}
\label{chap:confiabilidade_manutencao_do_sistema_produtivo}
Sempre há a probabilidade das coisas saírem erradas, todavia, em alguns casos é vital que os produtos ou serviços não falhem. Aceitar que falhas ocorrerão não é o mesmo que ignorá-las. É importante estudar ``como'' e ``por que'' as coisas falham para que possamos minimizar o surgimento destes erros.

A Confiabilidade é uma importante forma de estudar as falhas e mede a probabilidade de uma falha não ocorrer. Outras maneiras, como a taxa de falhas, que mede a frequência de acontecimento da falha e a disponibilidade, que é o período de tempo útil disponível para a operação também são usados frequentemente. A observação e ação em cima destes pontos fazem com que as atividades da operação reduzam drasticamente os seus pontos de falha.

Outro ponto a ser observado para manter a qualidade do processo é a manutenção. A mesma oferece uma lista de benefícios que contribuem diretamente com o aumento da confiabilidade dos produtos e da produção. Dividida principalmente em 3 modelos, a manutenção pode ser corretiva, a ação após o acontecimento da falha; preventiva, ação pré-acontecimento ou preditiva, ações de monitoramento visando a previsão de erros. Estas três estratégias podem ser usadas individualmente ou de forma combinada.

%------------------------------------------------------------------
\section{Sec1}
\label{sec:confiabilidade_manutencao_sec1}




%------------------------------------------------------------------
\section{Aplicação Prática}
\label{sec:confiabilidade_manutencao_aplicacao}
//todo Mostre como se faz uma análise de confiabilidade do sistema produtivo e como utilizar programas de manutenção.