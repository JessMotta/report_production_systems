\chapter{Confiabilidade e manutenção do sistema produtivo}
\label{chap:confiabilidade_manutencao_do_sistema_produtivo}
Sempre há a probabilidade das coisas saírem erradas, todavia, em alguns casos é vital que os produtos ou serviços não falhem. Aceitar que falhas ocorrerão não é o mesmo que ignorá-las. É importante estudar ``como'' e ``por que'' as coisas falham para que possamos minimizar o surgimento destes erros.

A Confiabilidade é uma importante forma de estudar as falhas e mede a probabilidade de uma falha não ocorrer. Outras maneiras, como a taxa de falhas, que mede a frequência de acontecimento da falha e a disponibilidade, que é o período de tempo útil disponível para a operação também são usados frequentemente. A observação e ação em cima destes pontos fazem com que as atividades da operação reduzam drasticamente os seus pontos de falha.

Outro ponto a ser observado para manter a qualidade do processo é a manutenção. A mesma oferece uma lista de benefícios que contribuem diretamente com o aumento da confiabilidade dos produtos e da produção. Dividida principalmente em três modelos, a manutenção pode ser corretiva, a ação após o acontecimento da falha; preventiva, ação pré-acontecimento ou preditiva, ações de monitoramento visando a previsão de erros. Estas três estratégias podem ser usadas individualmente ou de forma combinada.


%------------------------------------------------------------------
\section{Aplicação prática}
\label{sec:confiabilidade_manutencao_aplicacao}

O programa de manutenção empregado na Sunburn envolve os três principais modelos citados neste documento. A manutenção corretiva atua basicamente substituindo componentes e equipamentos danificados, além de reapertar as placas para a retirada dos chamados ``pontos quentes'', pontos que ultrapassam a temperatura máxima informada pelo fabricante. No programa preventivo, uma rotineira substituição do óleo e filtros lubrificantes dos geradores de emergência, coleta de óleo dos transformadores para análise cromatográfica e físico-química além da constante limpeza e reaperto das conexões das placas. A ação preditiva inclui a análise termográfica dos equipamentos de subestação, dos inversores e caixas de junção além de uma análise aérea dos módulos fotovoltaicos, nestes módulos são realizados testes periódicos de curva $I * V$. Outros testes de rotina são feitos nos transformadores de potência (elétricos) e nas ventilações forçadas (funcionamento).

Infelizmente, não foi possível obter os dados relacionados à análise de confiabilidade do processo produtivo desta empresa.