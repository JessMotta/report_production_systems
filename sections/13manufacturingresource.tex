\chapter{O \textit{Manufacturing Resource Planning}}
\label{chap:manufacturing_resource_planning}

Com base no \ac{PAP}, visto anteriormente, e nas análises de gerenciamento de estoques, pode-se utilizar a ferramenta \ac{MRP} para programar as atividades diárias da produção. Esta ferramenta permite o planejamento e controle, a curto prazo, dos meios de produção quando existe uma demanda dependente que consiste na definição das quantidades e momentos para a realização da compra dos insumos, bem como, na colocação das ordens de produção.

No âmbito do planejamento e controle dos meios de produção sob demanda dependente, o \textit{software} \ac{MRP} auxilia na execução das aquisições dos materiais diversificados e em diferentes quantidades dos diversos fornecedores, no planejamento da disponibilidade dos equipamentos necessários para cada etapa do processamento do pedido e na conciliação das duas atividades anteriores de forma que os produtos demandados possam ser produzidos de acordo com as datas estipuladas no contrato com o cliente. Além disso, esta ferramenta também atua de forma a otimizar o processamento do pedido de forma que o processo como um todo ocorra de forma contínua e eficiente.

Destaca-se que o \ac{MRP} utiliza o conceito de programação para trás, isto é, as tarefas necessárias para o atendimento do pedido são programadas de trás para a frente iniciando na data de entrega do pedido para o cliente. Deste modo, não há previsão de folgas para eventuais atrasos, contudo os estoques podem ser minimizados e os pagamentos para os fornecedores ocorrem no mais tardar possível. Para o correto funcionamento do \ac{MRP}, o mesmo necessita conhecer o Plano Mestre, isto é, o detalhamento de todos os pedidos firmados entre a empresa e seus clientes, a composição do produto com detalhamento técnico e, finalmente, os prazos de entrega dos fornecedores e os níveis de estoque da empresa.


%------------------------------------------------------------------
\section{Aplicação Prática}
\label{sec:manufacturing_resource_planning_aplicacao}
