\chapter{Conclusão}
\label{chap:conclusao}

Este trabalho apresentou um estudo das aplicações dos diversos conceitos relacionados aos Sistemas Produtivos Industriais na empresa SunBurn. A empresa SunBurn, nome fictício, é uma empresa geradora de energia elétrica localizada no nordeste do Brasil, em função dos elevados níveis de insolação, e que tem a concessionária de distribuição regional como seu único cliente. O processo produtivo desta empresa é caracterizado pela conversão de energia solar em energia elétrica através do modelo de produção contínuo com demanda dependente.

A empresa emprega técnicas relacionadas ao \ac{MRP} mas dentro da metodologia desenvolvida pelo \ac{SAP} para o seu controle de estoques de materiais utilizados na manutenção do processo produtivo. Além disso, apesar do processo de produção ser do tipo contínuo e diretamente entregue ao sistema de transporte, linhas de transmissão, com controle de geração baseado na demanda prevista com base em modelos e dados históricos, a produção da SunBurn não segue o modelo \textit{Just In Time}. Assim, a cadeia de suprimentos da mesma é centrada, à montante, no suprimento de equipamentos e materiais necessários para a manutenção da unidade produtiva e, à jusante, na venda do produto exclusivamente para o distribuidor regional.

Pode-se inferir através do processo de conversão de energia apresentado neste trabalho, que a correta manutenção (preventiva, preditiva e corretiva) dos equipamentos e estruturas do parque industrial influenciam fortemente na capacidade produtiva desta empresa. A falha de poucos equipamentos pode ocasionar uma paralisação de uma parte considerável do processo produtivo, ou até mesmo, sua completa estagnação. Por isso, a empresa apresenta um conjunto de processos relacionados ao controle do seu processo industrial e na manutenção do mesmo com o emprego das técnicas como Diagramas de Ishikawa, Diagramas de Correlação e \textit{Checklists}, dentre outras.

Finalmente, através das análises realizadas, acredita-se que o trabalho aqui apresentado consegue correlacionar os conteúdos estudos durante a disciplina de Sistemas Produtivos Industriais ministrada pelo Prof. Dr. Francisco Uchoa Passos durante o segundo semestre de 2020 junto ao SENAI CIMATEC com as práticas utilizadas por uma empresa existente e atuante no mercado brasileiro, ressalvados os dados e ou procedimentos que foram caracterizados como confidenciais pela empresa em questão.