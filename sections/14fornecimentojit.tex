\chapter{O fornecimento \textit{Just In Time}}
\label{chap:fornecimento_just_in_time}

Enquanto o \ac{MRP} pode ser entendido como uma ferramenta de utilizada no gerenciamento do processo produtivo, o fornecimento \ac{JIT} constitui uma nova filosofia de trabalho em contraponto com as práticas adotadas nas empresas industriais até o seu surgimento. Esta filosofia é definida por possuir as seguintes intenções: o produto deve ser fornecido ao cliente (interno ou externo) instantaneamente na quantidade estritamente especificada e esperada, evitar quaisquer tipo de ``desperdício'' de recursos produtivos, realizar as atividades de forma compartilhada em grupo com envolvimento efetivo das pessoas envolvidas no processo produtivo, bem como, a promoção continuada do aprimoramento das atividades executadas.

Com o intuito de atingir os objetivos prescritos de forma economicamente viável, mantendo a sua competitividade, os seguintes elementos devem ser observados:

\textbf{Baixos investimentos:} emprego criativo de pequenas máquinas e equipamentos universais, ou seja, que podem ser utilizadas na produção de diversos produtos.

\textbf{Flexibilidade:} por utilizar máquinas universais, o sistema produtivo \ac{JIT} apresenta naturalmente maior flexibilidade no conjunto de produtos possíveis de serem fabricados.

\textbf{Baixos custos de produção:} talvez o aspecto mais relevante nesta filosofia, mas certamente o que a faz ser tão competitiva com os modelos tradicionais de produção, é a redução dos custos de produção. Os baixos custos são provenientes das práticas adotadas de redução de desperdício, sendo que o conceito adotado de desperdício igual a toda e qualquer atividade que não agrega valor ao produto. Assim, atividades como transportes de mercadorias, geração e armazenamento de estoques, bem como movimentação de pessoas de forma desnecessária são classificadas como desperdícios. Além destes, a produção de defeitos, falhas e deficiências nos produtos ou processos também se enquadram nesse conceito.

\textbf{Fluxo contínuo:} de produtos e processos, sem interrupções, suave e rápido.

\textbf{Melhoria permanente:} a ideologia e mentalidade de que os produtos e processos podem ser constantemente aprimorados dever ser criada e mantida em todas as pessoas envolvidas no processo produtivos.

%------------------------------------------------------------------
\section{Aplicação Prática}
\label{sec:fonercimento_just_in_time_aplicacao}
