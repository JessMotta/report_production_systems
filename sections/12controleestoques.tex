\chapter{O CONTROLE DOS ESTOQUES}
\label{chap:controle_estoques}

Segundo \cite{slack2006administracao} o estoque é definido como todo e qualquer tipo de material acumulado na entrada de qualquer estágio do processo global de produção, ou de distribuição de produtos. Este acúmulo ocorre tanto quando o ritmo de processamento dos insumos for mais lento que o do estágio anterior como quando este ritmo for mais rápido, pois haverá esperas decorrentes das faltas momentâneas de materiais no estágio em questão.

Estoques entre quaisquer estágios do processo de produção e distribuição seriam indesejáveis e indicariam uma certa anomalia no processo, no que se refere à sincronia entre suas etapas. A seguir serão descritos alguns motivos pelos quais as empresas industriais optam por formar estoques.

\textbf{Incertezas da demanda:} Para o ambiente de demanda independente (incerta), as empresas fazem suas previsões de vendas e produz os produtos para atender essas previsões. Porém, tais previsões são frequentemente passíveis de incertezas, isto é, ora a demanda está acima da demanda real, ora abaixo. Para a primeira situação o resultado é a formação de estoques desnecessários, já para a segunda há falta de produtos para o cliente. Como, para todo negócio, a falta de produtos para o cliente deve ser evitada, a empresa deverá produzir além da previsão e consequentemente aumentará ainda mais a formação de estoques desnecessários, especialmente nas situações em que as previsões já estavam acima da demanda real.

\textbf{Incertezas da capacidade:} Esse motivo é baseado na possibilidade de ocorrer imprevistos como falhas dos equipamentos, das máquinas, falta de operadores e atrasos na entrega de matérias primas por parte dos fornecedores, que resultará na falta de produtos para o cliente. Por isso, as empresas costumam se prevenir contra todas estas incertezas através da formação de estoques de ``proteção''.

\textbf{Limitações impostas pelo próprio processo produtivo:} Alguns processos produtivos operam obrigatoriamente em determinadas quantidades mínimas, não sendo viável ou conveniente produzir-se em quantidades menores. Justamente por esta limitação que o processo produtivo, em alguns casos, poderá formar estoques, pois torna-se difícil conciliar a taxa de produção com a taxa de demanda do produto.

\textbf{Gargalos no processo produtivo:} Em algumas instalações industriais, é comum a existência de equipamentos do processo que têm uso universal, ou seja, por eles passam muitos dos modelos do mix de produtos. Estes equipamentos são os gargalos do processo produtivo e, consequentemente, causadores da formação de estoques intermediários de produtos em processamento. 

\textbf{Opção pela produção nivelada:} Quando as empresas optam por atender a sua demanda por intermédio de produção a nível constante, remanejando estoques, as quantidades excedentes de produtos produzidos nos períodos de baixa demanda servem para complementar a produção dos períodos de alta demanda. Esta decisão resulta na formação de estoques antecipados.

\textbf{Estoques no canal de distribuição:} O próprio canal de distribuição de um produto de consumo é um grande estoque de produtos em trânsito, que deve estar sempre preenchido, para que não falte produto para o cliente.


O cálculo destes estoques de segurança, do ponto de vista conceitual, basta adicionar uma quantidade suplementar ao lote de ressuprimento, quantidade esta suficiente para compensar os eventuais aumentos da demanda e/ou atrasos das entregas do material. Na área da produção, observa-se que alguns gerentes terminam por exagerar no dimensionamento do estoque de segurança, por não dispor de informações adequadas para o seu cálculo. Para auxiliar no dimensionamento adequado destas quantidades suplementares existem dois métodos frequentemente utilizados: O Lote Econômico de Compras (LEC) e a classificação ABC para os estoques.

%------------------------------------------------------------------
\section{Aplicação Prática}
\label{sec:controle_estoques_aplicacao}
A \textit{SunBurn}, para o controle de seus estoques, utiliza o sistema integrado de gestão empresarial \ac{SAP}. Um dos métodos deste sistema consiste em atribuir um limite mínimo para cada insumo, ou seja, toda vez que o insumo é consumido, é necessário registrar no sistema. Com isso, quando o determinado insumo atinge o limite mínimo atribuído para este, o sistema emite uma ordem de compra, evitando assim que o parque fique com déficit de algum material.

  

