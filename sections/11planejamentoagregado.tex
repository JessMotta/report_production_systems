\chapter{Planejamento agregado da produção}
\label{chap:planejamento_agregado}

Existem dois grupos de métodos, qualitativos e quantitativos, que são aplicados na previsão de demanda de curto, médio e longo prazo.
Planejamento e controle possuem uma forte correlação já que previsões podem conter muitas incertezas e controle ajuda a ajustar o plano.
O tempo aplicado para dimensionar curto, médio e longo prazo, depende de cada modelo de negócio. Por exemplo: hidrelétricas, longo prazo seriam décadas e no mundo da moda seriam no máximo um ano.

Para a unidade industrial tem-se a convenção:

\textbf{Longo prazo:} inclui-se futuros investimentos em novas instalações ou ampliações existentes.

\textbf{Médio prazo:} são as decisões tomadas ao longo de um ano.

\textbf{Curto prazo:} refere-se as atividades realizadas no dia a dia.

O planejamento agregado leva a resultados mais precisos do que ao realizar-se planejamento de forma detalhada.

Há duas formas de ajustar a capacidade à demanda de produção:

\textbf{Produção nivelada:} a empresa atende a demanda oscilante mantendo a produção constante ao longo do tempo. Dessa forma, independente das oscilações da demanda o nível da produção irá se manter constante, logo, a produção estará ``nivelada''.

\textbf{Produção acompanha a demanda:} o ajuste da capacidade é realizado por meio do acompanhamento da demanda de forma a flexibilizar a capacidade produtiva conforme as variações da demanda.


\section{Aplicação prática}
\label{sec:planejamento_agregado_aplicacao}

Tendo-se como base um circuito elétrico, a quantidade de energia entregue por uma fonte é igual à energia consumida pela carga mais a energia perdida através da transmissão da mesma entre a fonte e a carga. Assim, a empresa geradora (fonte) opera de forma que a produção acompanha a demanda do consumidor (carga). Este é um processo dinâmico e que deve ser controlado a fim de evitar falhas e problemas no sistema de geração, transmissão e distribuição. Por outro lado, existem as questões contratuais e de previsão de demanda futura que regem os planejamentos de produção da empresa ao longo do tempo.

% A SunBurn utiliza o método de produção nivelada pois independente das oscilações da demanda mantém-se a produção constante de energia elétrica. Há algumas alterações na produção, conforme o dia, devido a irradiação solar ocorrida.
