
\chapter{Projetos de novas instalações produtivas (localização, capacidade e rede de operações)}
\label{chap:projetos_de_novas}

Este capítulo auxiliará a tomar decisões a respeito de assuntos como localização das instalações produtivas e a capacidade adequada da instalação. %As questões sobre “onde”, “como” e “quanto” produzir serão respondidas de forma a tornar claros os estudos e métodos relacionados aos projetos citados nesta seção.
No fim, uma discussão a respeito de terceirização, verticalização e como decidir entre estas duas estratégias.

\section{Localização}
\label{sec:projeto_de_novas_localizacao}

A decisão para implementar uma unidade de produção em uma determinada localização está ligada a um conjunto de ``fatores de atração'' do local, que varia de negócio para negócio, de maneira que cada tipo de unidade de produção é influenciada a um certo conjunto destes fatores. Vale salientar que cada um dos fatores, individualmente, exerce sua própria força de atração, cuja intensidade, depende da influência específica do fator sobre o tipo de negócio.

Existem alguns métodos sistemáticos e quantitativos que podem ajudar no processo de tomada de decisão, dentre eles, a técnica de pontuação ponderada e o método do centro de gravidade que serão resumidos a seguir.

\textbf{Pontuação Ponderada:} Esse método envolve, primeiramente, o reconhecimento dos critérios que podem ser usados para avaliar as diversas possibilidades de localizações. Em seguida, é definido qual a importância relativa de cada um destes critérios e atribuído os fatores de ponderação (``pesos'') para cada um deles. Por fim, é feita uma avaliação de cada possibilidade de local em relação a cada um destes critérios.

\textbf{Centro de gravidade:} Esse método é baseado na ideia de que todas as localizações possuem um valor, que é a soma de todos os possíveis custos de transporte. A melhor localização é a que minimiza estes custos, que em uma analogia física, seria o centro de gravidade ponderado de todos os pontos de e para onde os bens são transportados. Com isso, podemos afirmar que o objetivo desta técnica é encontrar uma localização que minimize os custos de transporte.


\section{Cadeia de suprimentos: estrutura, verticalização e terceirização}
\label{sec:projetos_de_novas_supply_chain}
A cadeia de suprimentos de um processo produtivo é a relação da empresa com seus fornecedores e clientes, e a relação destes com seus fornecedores e clientes como descrita na Figura \ref{fig:supply_chain}. Nesta figura é possível perceber que os fornecedores que lidam diretamente com a operação são os de primeira camada, e os fornecedores dos fornecedores compõem a segunda camada, e estes fazem parte da montante do processo. Igualmente para o lado jusante, que tem os clientes de primeira camada, contato direto com a operação, e clientes dos clientes, que são os de segunda camada.
\par Além disso nota-se que fornecedores e clientes de primeira camada fazem parte da rede imediata de fornecimento, e a rede completa é chamada de rede total de suprimentos.


\begin{figure}[H]
    \centering
    \caption{Cadeia de Suprimentos (supply chain)}
    \includegraphics[width =\textwidth]{images/supply_chain.png}
    \caption*{Fonte: \cite{supplychain}}
    \label{fig:supply_chain}
\end{figure}

\par A importância de entender toda a rede é vital para a competitividade da empresa devido aos seguintes aspectos: identificar a relações imediatas, isso ajuda a conhecer melhor fornecedores e clientes; identificar elos significativos, saber quais partes da rede contribuem para alcançar os objetivos de desempenho valorizados pelos clientes finais, esta análise começa primeiramente pela parte da jusante e depois pela montante da rede a partir dos quais mais contribuem para o serviço do consumidor final, e por último, focar em questões de longo prazo, alguns elos dessa rede podem gerar situações como greves ou parada de máquina que ocasione uma interrupção no fluxo da operação, é importante estudar a possibilidade de ajudar ou substituir esse elo mais fraco.

\section{A capacidade adequada de uma instalação industrial}
A demanda de produção que será atendida pela fábrica depende de dois fatores: a demanda disponível no mercado e em segundo lugar, o capital disponível para investimento.

Para se reduzir os custos de produção a ponto de gerar uma economia de escala, é recomendado que a produção esteja sempre aproveitando a sua capacidade total. Esta prática confere vantagens como o aumento da participação da empresa no mercado, maximização das vendas, beneficio com os menores custos unitários. Entende-se como Custo unitário de produção o custo para se produzir uma unidade do produto, gerado pela divisão do custo total de produção de um lote pelo número de unidades do lote.

Na prática, este hábito citado não possui exatamente o efeito esperado. O uso constante da capacidade máxima da fabrica gera \textit{stress} nos equipamentos e trabalhadores e isto certamente causa aumentos inesperados nos custos de produção.

Conclue-se que a capacidade instalada de uma nova unidade de produção deve ser definida pelo seu projeto da capacidade, porém, a sua capacidade real é posteriormente calculada em meio a operação da fábrica, comparando seus custos com todos os volumes possíveis de produção.

\section{Verticalização x Terceirização}
É chamada de Verticalização, a estratégia que prevê a produção interna dos recursos que serão usados na fabricação de um produto ou na prestação de um serviço. Usando esta tática de gestão, toda a produção estará sob a inteira responsabilidade da própria empresa.

Por outro lado, nenhuma empresa faz tudo que é necessário para fabricar seus produtos sozinha. A forma de organização estrutural que permite a uma empresa transferir a outra suas atividades é chamada de Terceirização. Esta forma proporciona maior disponibilidade de recursos para sua atividade-fim, reduz a estrutura operacional, diminui custos, economiza recursos e desburocratiza a administração.

\par O suprimento interno ou terceirizado pode afetar diretamente os objetivos de desempenho de uma operação, sendo assim, cada estratégia oferece um pacote de vantagens e desvantagens, o que torna a decisão mais complexa e importante. Antes da escolha, é necessário sempre observar a importância estratégica da atividade.


\section{Aplicação Prática}
\label{sec:projetos_de_novas_aplicacao}
Para a SunBurn a escolha da localização onde seus parques de energia solar seriam implantados conformou com os dois fatores fundamentais para esse tipo de sistema produtivo: um grande espaço e estudo prévio durante dois anos para verificar o índice de irradiação solar naquele local.
\par Por esses motivos a reunião Nordeste foi escolhida para implementar os parques solares, já que esta dispõe dos dois elementos fundamentais.

\par A cadeia de suprimentos da SunBurn encontra-se descrita na Figura \ref{fig:cadeia_suprimentos_sunburn}. Nesta figura encontram-se definidas as relações da montante (fornecedores) e da jusante (cliente) com a operação produtiva. Também são identificados os fornecedores fixos e os sob cotação e demanda, além dos fluxos de serviço e de informação que existe entre cada elemento deste fluxograma.


\begin{figure}[H]
    \centering
    \caption{Cadeia de Suprimentos da SunBurn.}
    \includegraphics[width = \textwidth]{images/cadeia_suprimentos_sunburn.png}
    \caption*{Fonte: Autoria própria.}
    \label{fig:cadeia_suprimentos_sunburn}
\end{figure}
