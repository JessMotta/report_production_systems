\chapter{Tecnologia - Recurso essencial para a competitividade da empresa industrial}
\label{chap:tecnologia_recurso}

Tecnologia é o estudo sistemático sobre técnicas, processos, métodos e instrumentos de domínio da atividades humana. No setor industrial, este ofício representa o controle destas técnicas e também a capacidade de implementação de novas funcionalidades em máquinas para que estas possam operar de forma autônoma e sem o conhecimento do seu funcionamento interno. Em resumo, a tecnologia pode ser descrita como a arte de desenvolver e utilizar novas ferramentas.

É definida como Tecnologia do Produto aquela introduzida no produto e usada na criação do seu projeto ou no decorrer do seu desenvolvimento. De acordo com o Manual de Apoio ao Preenchimento da Pintec (2014) %!tem que citar aqui
, a inovação de produto consiste em criar modificações nos atributos dos bens ou serviços, tais como mudanças operacionais e na forma como ele é percebido pelos consumidores.

A tecnologia empregada na rotina de produção de um produto, destinada a fazer com que sejam atingidas as metas estabelecidas pela empresa é chamada Tecnologia de Processo. O tipo de produto a ser produzido é quem dita a tecnologia usada no processo industrial, ao lado da quantidade e volume que serão fabricados.

É exemplo de tecnologia de processos a empregada na produção de materiais. O uso de robôs e outras máquinas vem crescendo e se consolidando como algo indispensável. O uso de máquinas de corte, furadeiras e \ac{CNC} complementam a tendencia no uso da automação.

O processamento de informações, assim como a \ac{TI}, vem avançando e se popularizando exponencialmente. Tem a intenção de ser abrangente e atua como técnologia principal ou auxiliar em quase todos os ramos de trabalho.

O emprego das tecnologias permitem a constante inovação de produtos e processos e garantem a presença competitiva no mercado. Produtos novos, modificados e mais eficientes aumentam a procura e melhoram a opinião do cliente a respeito da empresa.


\section{Aplicação Prática}
\label{sec:tecnologia_recurso_aplicacao}

//todo Considere o projeto de lançamento de um novo antibiótico por parte de um Laboratório Farmacêutico internacional, com sede na Suíça e com
uma unidade de produção no Brasil. Para o referido projeto, tente descrever:
a) como a empresa emprega a tecnologia do produto para desenvolver o
antibiótico na Suíça;
b) como a empresa emprega a tecnologia do processo para fabricar o
antibiótico no Brasil.